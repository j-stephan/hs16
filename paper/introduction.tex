\section{Introduction}\label{intro}

With the release of the C++11 standard in 2011 the C++ community was introduced to several new language and library features~\cite{cpp11std}. Together with a new programming philosophy (see Section~\ref{intro:philosophy}) these changes were so extensive that the original creator of the C++ programming language, Bjarne Stroustrup, felt like it was a ``new language''~\cite{tcpp}.

The goals of this report are
\begin{itemize}
\item to present modern C++ features relevant in a context of HPC and performance analysis,
\item to evaluate and analyse these feature's influence on the aforementioned context,
\item to implement example programs making use of these features and their analysis with Score-P~\cite{scorep} and Vampir~\cite{vampir} and finally
\item to develop recommendations for enabling Score-P and Vampir to support those features they currently can not instrument correctly.
\end{itemize}

\subsection{C++11 and C++14}\label{intro:cpp11_14}

The C++11 standard~\cite{cpp11std} was published in 2011, thirteen years after the last major standard C++98 and eight years after the last minor standard C++03. While it maintained near-perfect backward compatibility~\cite{cppfaq_learn} it introduced a lot of new features to the core language and the standard library.

The current C++14 standard~\cite{cpp14std} brought fewer changes to both the language and the library when compared to its predecessor and can be considered ``bugfix'' to C++11. Because of this the contents of the sections below are focused on C++11, mentioning C++14 where changes to the C++11 standard can be applied.

\subsection{The Modern C++ Philosophy}\label{intro:philosophy}

Apart from language standardisation the Standard C++ Foundation is also advertises a new programming philosophy, suggesting more abstraction, using the \gls{raii} pattern and relying on the \gls{stl} in almost all situations. In support of this cause Bjarne Stroustrup (creator of C++) and Herb Sutter (president of the Standard C++ Foundation) are working on the detailed \textit{C++ Core Guidelines}, the current draft being publicly available~\cite{cpp_guidelines}. The code examples in this document are following these guidelines wherever possible.
\begin{appendix}

\section{Score-P / Vampir Test Cases}\label{app:scorep}

\subsection{Concurrency}\label{app:scorep_conc}

\subsubsection{\texttt{std::async} -- Default Behaviour}\label{app:scorep_conc_async_default}

\begin{minted}[fontsize=\small]{c++}
#include <algorithm>
#include <cstddef>
#include <cstdint>
#include <future>
#include <iterator>
#include <numeric>
#include <vector>

constexpr auto vec_size = std::size_t{1000000};

auto task1() -> void
{
    // fill a vector with values (vec_size, 0] and sort it
    auto vec = std::vector<std::int32_t>{};
    vec.resize(vec_size);
    std::iota(std::rbegin(vec), std::rend(vec), 0);
    std::sort(std::begin(vec), std::end(vec));
}

auto task2() -> std::int32_t
{
    // fill a vector with random values and return the maximum
    auto vec = std::vector<std::int32_t>{};
    vec.resize(vec_size);
    std::generate(std::begin(vec), std::end(vec), std::rand);
    auto it = std::max_element(std::begin(vec), std::end(vec));
    return *it;
}

auto main() -> int
{
    auto f1 = std::async(task1);
    auto f2 = std::async(task2);
    
    f1.get(); // task1 has return type void
    auto res = f2.get();
    
    return 0;
}
\end{minted}

\subsubsection{\texttt{std::async} -- Asynchronous Evaluation}\label{app:scorep_conc_async_async}

\begin{minted}[fontsize=\small]{c++}
#include <algorithm>
#include <cstddef>
#include <cstdint>
#include <future>
#include <iterator>
#include <numeric>
#include <vector>

constexpr auto vec_size = std::size_t{1000000};

auto task1() -> void
{
    // fill a vector with values (vec_size, 0] and sort it
    auto vec = std::vector<std::int32_t>{};
    vec.resize(vec_size);
    std::iota(std::rbegin(vec), std::rend(vec), 0);
    std::sort(std::begin(vec), std::end(vec));
}

auto task2() -> std::int32_t
{
    // fill a vector with random values and return the maximum
    auto vec = std::vector<std::int32_t>{};
    vec.resize(vec_size);
    std::generate(std::begin(vec), std::end(vec), std::rand);
    auto it = std::max_element(std::begin(vec), std::end(vec));
    return *it;
}

auto main() -> int
{
    auto f1 = std::async(std::launch::async, task1);
    auto f2 = std::async(std::launch::async, task2);
    
    f1.get(); // task1 has return type void
    auto res = f2.get();
    
    return 0;
}
\end{minted}

\subsubsection{\texttt{std::async} -- Lazy Evaluation}\label{app:scorep_conc_async_lazy}

\begin{minted}[fontsize=\small]{c++}
#include <algorithm>
#include <cstddef>
#include <cstdint>
#include <future>
#include <iterator>
#include <numeric>
#include <vector>

constexpr auto vec_size = std::size_t{1000000};

auto task1() -> void
{
    // fill a vector with values (vec_size, 0] and sort it
    auto vec = std::vector<std::int32_t>{};
    vec.resize(vec_size);
    std::iota(std::rbegin(vec), std::rend(vec), 0);
    std::sort(std::begin(vec), std::end(vec));
}

auto task2() -> std::int32_t
{
    // fill a vector with random values and return the maximum
    auto vec = std::vector<std::int32_t>{};
    vec.resize(vec_size);
    std::generate(std::begin(vec), std::end(vec), std::rand);
    auto it = std::max_element(std::begin(vec), std::end(vec));
    return *it;
}

auto main() -> int
{
    auto f1 = std::async(std::launch::deferred, task1);
    auto f2 = std::async(std::launch::deferred, task2);
    
    f1.get(); // task1 has return type void
    auto res = f2.get();
    
    return 0;
}
\end{minted}

\subsubsection{\texttt{std::thread} -- Spawn and Join}\label{app:conc_thread_join}

\begin{minted}[fontsize=\small]{c++}
#include <algorithm>
#include <cstddef>
#include <cstdint>
#include <iterator>
#include <numeric>
#include <thread>
#include <vector>

constexpr auto vec_size = std::size_t{1000000};

auto f() -> void
{
    auto vec = std::vector<std::int32_t>{vec_size};
    std::iota(std::rbegin(vec), std::rend(vec), 0);
    std::sort(std::begin(vec), std::end(vec));
}

auto main() -> int
{
    auto t = std::thread{f};
    t.join();
    
    return 0;
}
\end{minted}

\subsubsection{\texttt{std::thread} -- Spawn and Detach}\label{app:conc_thread_detach}

\begin{minted}[fontsize=\small]{c++}
#include <algorithm>
#include <chrono>
#include <cstddef>
#include <cstdint>
#include <iterator>
#include <numeric>
#include <thread>
#include <vector>

constexpr auto vec_size = std::size_t{1000000};

auto f() -> void
{
    auto vec = std::vector<std::int32_t>{vec_size};
    std::iota(std::rbegin(vec), std::rend(vec), 0);
    std::sort(std::begin(vec), std::end(vec));
}

auto main() -> int
{
    auto t = std::thread{f};
    t.detach();
    
    using namespace std::chrono_literals;
    std::this_thread::sleep_for(5s);
    
    return 0;
}
\end{minted}

\subsection{Synchronization}\label{app:scorep_sync}

\subsubsection{\texttt{std::mutex}}\label{app:scorep_sync_mutex}

\begin{minted}[fontsize=\small]{c++}
#include <iostream>
#include <mutex>

#include <pthread.h>
#include <unistd.h>

std::mutex m;

auto f1(void*) -> void*
{
    std::lock_guard<std::mutex> l(m);
    std::cout << "Mutex locked in f1" << std::endl;
    usleep(20 * 1000);
    return nullptr;
}

auto f2(void*) -> void*
{
    usleep(10 * 1000);
    std::lock_guard<std::mutex> l(m);
    std::cout << "Mutex locked in f2" << std::endl;
    usleep(20 * 1000);
    return nullptr;
}

auto main() -> int
{
    pthread_t threads[2];
    
    pthread_create(&threads[0], nullptr, f1, 0);
    pthread_create(&threads[1], nullptr, f2, 0);
    
    pthread_join(threads[0], nullptr);
    pthread_join(threads[1], nullptr);
    
    return 0;
}
\end{minted}

\subsubsection{\texttt{std::condition\_variable}}\label{app:scorep_sync_cv}

\begin{minted}[breaklines,breakafter=\,,fontsize=\small]{c++}
#include <algorithm>
#include <array>
#include <condition_variable>
#include <iostream>
#include <iterator>
#include <mutex>

#include <pthread.h>
#include <unistd.h>

std::condition_variable cv;
std::mutex m;

auto done = false;

auto wait(void*) -> void*
{
    std::unique_lock<std::mutex> l(m);
    std::cout << "Waiting...\n";
    cv.wait(l, []{ return done; });
    std::cout << "...finished waiting. done == true\n";
    return nullptr;
}

auto signal(void*) -> void*
{
}

auto main() -> int
{
    auto threads = std::array<pthread_t, 4>{};
    std::for_each(std::begin(threads), std::begin(threads) + 3, [](auto& t){ pthread_create(&t, nullptr, wait, 0); });
    pthread_create(&threads[3], nullptr, signal, 0);
    
    for(auto&& t : threads)
        pthread_join(t, nullptr);
        
    return 0;
}
\end{minted}

\subsection{File I/O}

\subsubsection{\texttt{std::fstream}}\label{app:scorep_fstream}

\begin{minted}[breaklines,breakafter=\,,fontsize=\small]{c++}
#include <algorithm>
#include <cstdlib>
#include <fstream>
#include <iostream>
#include <string>
#include <vector>

auto main() -> int
{
    auto path = std::string{"test.bin"};
    auto output = std::vector<int>{};
    auto input = std::vector<int>{};
    output.resize(1000);
    input.resize(1000);
    
    std::generate(std::begin(output), std::end(output), std::rand);
    
    {
        std::ofstream out{path.c_str(), std::ios::binary};
        out.write(reinterpret_cast<const char*>(output.data()), output.size() * sizeof(int));
    }
    {
        std::ifstream in{path.c_str(), std::ios::binary};
        in.read(reinterpret_cast<char*>(input.data()), input.size() * sizeof(int));
    }
    
    if(!std::equal(std::begin(output), std::end(output), std::begin(input)))
        std::cerr << "Error: Input does not match output." << std::endl;
        
    return 0;
}
\end{minted}

\subsubsection{\texttt{C-style I/O}}\label{app:scorep_c_io}

\begin{minted}[breaklines,breakafter=\,,fontsize=\small]{c++}
#include <algorithm>
#include <cstdio>
#include <cstdlib>
#include <iostream>
#include <string>
#include <vector>

auto main() -> int
{
    auto path = std::string{"test.bin"};
    auto output = std::vector<int>{};
    auto input = std::vector<int>{};
    output.resize(1000);
    input.resize(1000);
    
    std::generate(std::begin(output), std::end(output), std::rand);
    
    auto out = std::fopen(path.c_str(), "wb");
    std::fwrite(output.data(), sizeof(int), output.size(), out);
    std::fclose(out);
    
    auto in = std::fopen(path.c_str(), "rb");
    std::fread(input.data(), sizeof(int), input.size(), in);
    std::fclose(in);
    
    if(!std::equal(std::begin(output), std::end(output), std::begin(input)))
        std::cerr << "Error: Input does not match output." << std::endl;
        
    return 0;
}
\end{minted}

\end{appendix}